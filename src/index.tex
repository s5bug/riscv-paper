\maketitle

\section{What is RISC-V?}

RISC-V is a general-purpose computing architecture that is meant to be small and
complete, without tying itself to hardware implementation details
\cite{waterman2019risc}.

\subsection{Reduced Instruction Set Computing}

Many modern instruction sets in use, such as x86, continue to expand their
instruction set, leading to CPUs being harder to build, compact, and verify.
Reduced Instruction Set Computers (called RISCs) aim to solve this problem
by using instruction sets that make it easier to create CPUs and offer internal
optimizations.

The RISC philosophy has been in use since Alan Turing's Automatic Computing
Engine \cite{doran2005computer}. Modern RISC machines are usually characterized
by a large amount of registers, and few ways to interact with memory. For
example, RISC-V has 32 registers, and the only load/store instructions do not
modify what's in memory \cite{waterman2019risc}.

\section{Comparison to Hack and Hack VM}

The Hack computer also has very few instructions, the A-instruction and the
C-instruction. If "RISC" is taken literally, Hack could be said to be a RISC
machine. However, Hack is more classified as a MISC, or Minimal Instruction Set
Computer. This is because Hack packs a lot of functionality into a singular
C-instruction. In comparison, RISC-V has separate instructions for each type of
register comparison and jump \cite{waterman2019risc}.

The Hack instruction set is directly tied to the Hack CPU: Instructions assume
that both ROM and RAM are exactly addressable by 16 byte addresses. Expressions
in C-instructions are directly representative of how the ALU works internally.
Instructions are always meant to take one and only one clock cycle. On the
contrary, RISC-V is not tied to a hardware implementation: instructions are
very general as to not enforce a design that isn't portable. Only a minimal
amount of integer operations are required for the basic instruction set: more
complex operations for multiplication and floats are extensions, and can be
implemented in software if they do not exist on the processor
\cite{waterman2019risc}.

\subsection{Word Size}

Hack uses a 16-bit word. Hack also only has a 16-bit-addressable ROM and a
16-bit-addressable RAM. Therefore accessing memory in Hack will always
succeed, and there is no memory that is unaccessible. RISC-V has three
word-size specifications. 32-bit words and 64-bit words are both ratified in
the specification, while 128-bit words are still open. RISC-V, because it is
not tied to a hardware implementation, can have vacant or nonaddressable memory
regions. When it encounters them, it raises an exception in the processor.

RISC-V aims to be usable both as a bare-metal architecture as well as with
operating systems. To accomplish this, it allows memory mapping and address
translation. Hack doesn't load the program data into memory, it has a split ROM
and RAM, so it cannot truly run an operating system.

\subsection{Registers}

Hack only has three accessible registers: Data, Memory, and Address. The
Address and Memory registers have side effects when stored to, so they are not
considered general-purpose registers. As such, Hack can be said to only have
one general-purpose register. RISC-V defines 32 registers, 27 of which can be
used as general-purpose registers under the RISC-V calling convention.

\begin{figure}
    \begin{verbatim}
        // Add 4 to D
        @4
        D=D+A
        // Address D
        A=D
        // Set to 0
        M=0
    \end{verbatim}
    \caption{Hack ASM to store 0 to D+4}
    \label{hack-store}
\end{figure}

\begin{figure}
    \begin{verbatim}
        ; store zero to a0+4
        sw zero,4(a0)
    \end{verbatim}
    \caption{RISC-V ASM to store 0 to a0+4}
    \label{riscv-store}
\end{figure}

Hack uses the Memory register to interface with RAM directly, while RISC-V can
only store to RAM with dedicated store instructions, and can only load from
memory with load instructions or relative addressing modes. Shown in
Figure~\ref{hack-store} is the Hack Assembly code that would store 0 to an
offset of a memory address in D. We can see from this that the original
contents of the D register and the A register are lost after this operation.
Shown in Figure~\ref{riscv-store} is the equivalent RISC-V code, which does not
clobber the value in a0, because memory is not accessed via a register.

\subsection{Stack Operations}

In both Hack and RISC-V, the stack pointer is manually controlled. However, in
Hack, the stack pointer lives at a memory address, while in RISC-V, the stack
pointer is a register. This makes it a lot harder to dereference the stack
pointer in Hack, as Hack only has one addressing mode, while RISC-V can use its
relative addressing mode. However, interacting with memory in Hack is done via
a register, so doing arithmetic at the top of the stack takes less
instructions.

\begin{figure}
    \begin{verbatim}
        // Pop the value at the top of the stack into D
        @SP
        M=M-1
        A=M
        D=M
        // Address the value at the top of the stack
        @SP
        A=M
        // Add D to it
        M=M+D
    \end{verbatim}
    \caption{Hack ASM to add two values at the top of the stack}
    \label{hack-add}
\end{figure}

\begin{figure}
    \begin{verbatim}
        ; pop a value from the stack and load into a0
        addiu sp,sp,4
        lw a0,(sp)
        ; pop a value from the stack and load into a1
        lw a1,(sp)
        ; a0 += a1
        addu a0,a0,a1
        ; store a0 back to the top of the stack
        sw a0,(sp)
    \end{verbatim}
    \caption{RISC-V32I ASM to add two values at the top of the stack}
    \label{riscv-add}
\end{figure}

In the Forth-like VM language, arithmetic operations are implemented as stack
modifications, because the VM language repeatedly clobbers A and M, so math
cannot be done with them. In RISC-V, there is more than one general-purpose
register, so math can be done without touching the stack. However, many of the
Hack VM operations can be made much more readable in RISC-V ASM.
Figure~\ref{hack-add} shows the implementation of \verb|add| in Hack ASM, and
Figure~\ref{riscv-add} shows the equivalent RISC-V (32-bit) assembly.

\section{Applications of RISC-V}

RISC-V is already being used in embedded devices. Companies like SiFive develop
Systems on Chips that run RISC-V for usage in Artificial Intelligence and
Machine Learning \cite{openfive2021soc}. There are also consumer computers that
run RISC-V being developed, such as the BeagleV. RISC-V is supposed to be
easier for developers, too. It introduces its own secure boot interface for
operating systems, called OpenSBI \cite{riscv2021opensbi}. Under discussion are
certain debugging features for the architecture, such as being able to feed
individual instructions and inspect output over some communication medium like
USB \cite{riscv2021debugging}.

Perhaps the most exciting development will be in home computing: the design of
RISC-V encourage highly-pipelined processors that are small and efficient
enough to increase clock speed and core count without highly increasing cost.
However, this opportunity hasn't been well-explored yet, and the lack of RISC-V
home computers means that software has not been ported to it.
